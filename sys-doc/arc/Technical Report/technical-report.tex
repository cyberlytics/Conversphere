\documentclass[conference]{IEEEtran}
\IEEEoverridecommandlockouts
% The preceding line is only needed to identify funding in the first footnote. If that is unneeded, please comment it out.
\usepackage{cite}
\usepackage{amsmath,amssymb,amsfonts}
\usepackage{algorithmic}
\usepackage{graphicx}
\usepackage{textcomp}
\usepackage{xcolor}
\usepackage{hyperref}
\def\BibTeX{{\rm B\kern-.05em{\sc i\kern-.025em b}\kern-.08em
    T\kern-.1667em\lower.7ex\hbox{E}\kern-.125emX}}

\hypersetup{      
	urlcolor=blue,
}
\begin{document}

	\title{Conversphere}
	\author{
		\IEEEauthorblockN{1\textsuperscript{st} Feil Lukas}
		\IEEEauthorblockA{
			\textit{OTH Amberg-Weiden} \\
			Bad Kötzting, Deutschland \\
			l.feil@oth-aw.de
		}
		\and
		\IEEEauthorblockN{2\textsuperscript{nd} Markus Fleischmann}
		\IEEEauthorblockA{
			\textit{OTH Amberg-Weiden} \\
			Wernberg-Köblitz, Deutschland\\
			m.fleischmann2@oth-aw.de
		}
		\and
		\IEEEauthorblockN{3\textsuperscript{rd} Stefan Reger}
		\IEEEauthorblockA{
			\textit{OTH Amberg-Weiden} \\
			Schwarzenfeld, Deutschland \\
			s.reger@oth-aw.de
		}
		\and
		\IEEEauthorblockN{4\textsuperscript{th} Lukas Rupp}
		\IEEEauthorblockA{
			\textit{OTH Amberg-Weiden} \\
			Amberg, Deutschland \\
			l.rupp@oth-aw.de
		}
		\and
		\IEEEauthorblockN{5\textsuperscript{th} Sorel Tahata Djoumsi}
		\IEEEauthorblockA{
			\textit{OTH Amberg-Weiden} \\
			Regensburg, Deutschland \\
			s.tahata-djoumsi@oth-aw.de
		}
	}

	\maketitle

	\begin{abstract}
		Web Chat-App mit Raumfunktion und Proximity-Chat
	\end{abstract}

	\begin{IEEEkeywords}
		PWA, Chat, Proximity, Room, Web App, Angular, NodeJS, MongoDB, Express
	\end{IEEEkeywords}

	\section{Problemstellung}
    \subsection{Mission statement}
	Conversphere ist eine Web Chat-Anwendung, die es Benutzern ermöglicht, miteinander über Textnachrichten zu kommunizieren. 
    Dadurch wird ein sicherer Raum geschaffen in welchem man seine Gedanken und Ideen teilen kann und gleichzeitig sozial distanziert bleibt. 
    Benötigt wird dafür lediglich ein Gerät mit Webbrowser und ein Internetzugang.
	\ \\

	\subsection{Kontextabgrenzung}
	Das System wird im folgenden als Blackbox abgespeichert. 
    Alle verwendeten externen Systeme werden als Box um die Anwendung „Conversphere“ dargestellt. 
    Der menschliche Akteur als „Spieler“ kann in diesem Fall als Plural aufgefasst werden, da zur Kommunikation als Dialog mindestens zwei Spieler benötigt werden.
    - Bild
    -	Für die Entwicklung des Frontend werden „Angular“ und deren Bibliotheken aus „Angular-Material“ verwendet Angular Progressiv Web App. 
    Die Kommunikation zum Backend findet über „Socket.io“ statt. Zur Speicherung der Nutzer-, Spiel- und Chatdaten wird die NoSQL Datenbank „MongoDB“ verwendet. 
    Das Backend läuft auf einem „Express.js“-Server welcher auf „Node.js“ basiert. 
    Die Website läuft als static Website über „AWS S3“ und wird mittels einem „Docker Containers“ in AWS ECS gehostet.
	\ \\

	\subsection{Qualitätsziele}
	\begin{tabular}{ | | | | } 
        \hline
        Gute Benutzbarkeit & Conversphere ist für Benutzer aller Altersklassen intuitiv und ohne Erklärung bedienbar.  \\ 
        \hline
        Hohe Zuverlässigkeit & Das System steht den Spielern jederzeit zur Verfügung und stellt Chaträume live dar. \\ 
        \hline
        Gute Wartbarkeit & Conversphere ist ohne großen Administrationsaufwand betreibbar und leicht um zusätzliche Features erweiterbar. \\ 
        \hline
      \end{tabular}
	\ \\

	\subsection{Entscheidende Rahmenbedingungen}
	\begin{tabular}{ | | | | } 
        \hline
        Zeitlicher Rahmen & Das Projekt ist innerhalb von fünf Kalenderwochen abzuschließen.  \\ 
        \hline
        Technischer Anspruch & Das Projekt muss mind. einen der Themenbereiche Web-Anwendung. \\ 
        \hline
        Einfachheit in der Anwendung & Die Anwendung soll so einfach gehalten werden, dass sie im gegebenen Zeitraum umgesetzt werden kann. \\ 
        \hline
      \end{tabular}
	\ \\

	\section{Lösungsstrategien}
	\subsection{Lösungsansätze}
	Die zuvor benannten Qualitätsziele wurden im Projekt Conversphere folgendermaßen berücksichtigt:  \\
    \begin{tabular}{ | | | | } 
        \hline
        Gute Benutzbarkeit & Durch eine simpel gehaltene Angular-Oberfläche kann die Anwendung intuitiv und ohne Erklärung benutzt werden. 
        Die Oberfläche ist selbsterklärend und kann ansonsten bei der Anwendung weiter erkundet werden.
        Zur Benutzung ist lediglich ein Webserver nötig.  \\ 
        \hline
        Hohe Zuverlässigkeit & Da die Infrastruktur auf AWS-Diensten basiert ist das System praktisch ohne zu erwartende Ausfälle dauerhaft verfügbar.
        Nur durch die Benutzereigene Internetanbindung können Probleme bei der Benutzung auftreten.  \\ 
        \hline
        Gute Wartbarkeit & Die Nutzung . \\ 
        \hline
      \end{tabular}


    \subsection{Technologiestack}
    Das Frontend von Conversphere wird durch Angular gegeben.
    Die Backend-Infrastruktur basiert auf einem Express-Server welcher auf Node.js läuft. Zur Kommunikation zwischen Frontend und Backend wird eine Socket.io Verbindung genutzt.
    Alle Nutzer-, Spiel- und Chatdaten werden auf der NoSQL Datenbank „MongoDB“ gespeichert.
    Für die Verwaltung der AWS Infrastruktur wird AWS S3 benutzt und über AWS ECS gehostet.
    
	

\end{document}