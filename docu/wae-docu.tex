\documentclass[conference]{IEEEtran}
\IEEEoverridecommandlockouts
% The preceding line is only needed to identify funding in the first footnote. If that is unneeded, please comment it out.
\usepackage{cite}
\usepackage{amsmath,amssymb,amsfonts}
\usepackage{algorithmic}
\usepackage{graphicx}
\usepackage{textcomp}
\usepackage{xcolor}
\def\BibTeX{{\rm B\kern-.05em{\sc i\kern-.025em b}\kern-.08em
    T\kern-.1667em\lower.7ex\hbox{E}\kern-.125emX}}
\begin{document}

\title{Conversphere\\
}
\author{\IEEEauthorblockN{1\textsuperscript{nd} Feil Lukas}
\IEEEauthorblockA{\textit{Oth Amberg Weiden} \\
Bad Kötzting, Deutschland \\
l.feil@oth-aw.de}
\and
\IEEEauthorblockN{2\textsuperscript{nd} Given Name Surname}
\IEEEauthorblockA{\textit{dept. name of organization (of Aff.)} \\
\textit{name of organization (of Aff.)}\\
City, Country \\
email address or ORCID}
\and
\IEEEauthorblockN{3\textsuperscript{rd} Given Name Surname}
\IEEEauthorblockA{\textit{dept. name of organization (of Aff.)} \\
\textit{name of organization (of Aff.)}\\
City, Country \\
email address or ORCID}
\and
\IEEEauthorblockN{4\textsuperscript{th} Given Name Surname}
\IEEEauthorblockA{\textit{dept. name of organization (of Aff.)} \\
\textit{name of organization (of Aff.)}\\
City, Country \\
email address or ORCID}
\and
\IEEEauthorblockN{5\textsuperscript{th} Given Name Surname}
\IEEEauthorblockA{\textit{dept. name of organization (of Aff.)} \\
\textit{name of organization (of Aff.)}\\
City, Country \\
email address or ORCID}
\and
\IEEEauthorblockN{6\textsuperscript{th} Given Name Surname}
\IEEEauthorblockA{\textit{dept. name of organization (of Aff.)} \\
\textit{name of organization (of Aff.)}\\
City, Country \\
email address or ORCID}
}

\maketitle

\begin{abstract}
Kurzbeschreibung
\end{abstract}

\begin{IEEEkeywords}
Schlüsselwörter
\end{IEEEkeywords}

\section{Einführung}
TEXT
\ \\ %Erzeugt ein Leerzeichen und beendet dann die Zeile, also Leerzeile

\section{Motivation}
\ \\

\section{Verwandte Arbeiten}
\subsection{Erste Subsection}
TEXT
\ \\

\section{Alleinstellungmerkmal}
\ \\

\section{Technisches Grundkonzept}
\subsection{Datenbank - Mongo DB}
Zur Speicherung von Nutzer und Spiel bzw. Chatdaten soll eine NoSQL MongoDB Datenbank zum Einsatz kommen. 
MontoDB Atlas bietet dafür einen kostenlosen Service, welcher eine gehostete MongoDB zu Verfügung stellt.

\subsection{Backend - Express.js}
Das Backend soll auf einem in Node.js laufenden Express.js Server basieren. 
Dieser soll die gesammte Spiellogik implementieren und alle anfallenden Daten zur speicherung in die Datenbank leiten.

\subsection{Schnittstellen - REST API}
Der Express Server sowie das Frontend sollen über eine REST API miteinander Kommunizieren.
Es sollen dabei für jede Funktion bzw. jede Art von Daten eine eigene Domains sowohl im Frontend als auch im Backend erstellt werden.

\subsection{Frontend - Angular PWA}
Für das Frontend soll mit Hilfe von Anuglar eine Progressiv Web App erstellt werden welche mit Hilfe einer Render bzw. Animierungsbiliothek erweitert wird.

\subsection{Deployment - AWS S3 und EC2}
Die Webseite soll als static Website über AWS S3 an die Clienten verteilt werden.
Das Backend soll mit Hilfe eines Docker Containers auf einem AWS EC2 gehostet werden.

\section{Anforderungen}
\ \\














\newpage
\quad % without \quad no new page would be insert!!!
\newpage

\section{Beispiele}
Text6

Text 7

\subsection{Erste Subsection}\label{AA}
Define abbreviations and acronyms the first time they are used in the text, 
even after they have been defined in the abstract. Abbreviations such as 
IEEE, SI, MKS, CGS, ac, dc, and rms do not have to be defined. Do not use 
abbreviations in the title or heads unless they are unavoidable.

\subsection{Units}
\begin{itemize}
\item Item
\item Item 2

\end{itemize}

\subsection{Equations}
Rechnung
\begin{equation}
a+b=\gamma\label{eq}
\end{equation}

Be sure that the 
symbols in your equation have been defined before or immediately following 
the equation. Use ``\eqref{eq}'', not ``Eq.~\eqref{eq}'' or ``equation \eqref{eq}'', except at 
the beginning of a sentence: ``Equation \eqref{eq} is . . .''


\subsection{Figures and Tables}
\paragraph{Positioning Figures and Tables} Place figures and tables at the top and 
bottom of columns. Avoid placing them in the middle of columns. Large 
figures and tables may span across both columns. Figure captions should be 
below the figures; table heads should appear above the tables. Insert 
figures and tables after they are cited in the text. Use the abbreviation 
``Fig.~\ref{fig}'', even at the beginning of a sentence.

\begin{table}[htbp]
\caption{Table Type Styles}
\begin{center}
\begin{tabular}{|c|c|c|c|}
\hline
\textbf{Table}&\multicolumn{3}{|c|}{\textbf{Table Column Head}} \\
\cline{2-4} 
\textbf{Head} & \textbf{\textit{Table column subhead}}& \textbf{\textit{Subhead}}& \textbf{\textit{Subhead}} \\
\hline
copy& More table copy$^{\mathrm{a}}$& &  \\
\hline
\multicolumn{4}{l}{$^{\mathrm{a}}$Sample of a Table footnote.}
\end{tabular}
\label{tab1}
\end{center}
\end{table}

\begin{figure}[htbp]

\caption{Example of a figure caption.}
\label{fig}
\end{figure}


\begin{thebibliography}{00}
\bibitem{b1} G. Eason, B. Noble, and I. N. Sneddon, ``On certain integrals of Lipschitz-Hankel type involving products of Bessel functions,'' Phil. Trans. Roy. Soc. London, vol. A247, pp. 529--551, April 1955.
\bibitem{b2} J. Clerk Maxwell, A Treatise on Electricity and Magnetism, 3rd ed., vol. 2. Oxford: Clarendon, 1892, pp.68--73.
\bibitem{b3} I. S. Jacobs and C. P. Bean, ``Fine particles, thin films and exchange anisotropy,'' in Magnetism, vol. III, G. T. Rado and H. Suhl, Eds. New York: Academic, 1963, pp. 271--350.
\bibitem{b4} K. Elissa, ``Title of paper if known,'' unpublished.
\bibitem{b5} R. Nicole, ``Title of paper with only first word capitalized,'' J. Name Stand. Abbrev., in press.
\bibitem{b6} Y. Yorozu, M. Hirano, K. Oka, and Y. Tagawa, ``Electron spectroscopy studies on magneto-optical media and plastic substrate interface,'' IEEE Transl. J. Magn. Japan, vol. 2, pp. 740--741, August 1987 [Digests 9th Annual Conf. Magnetics Japan, p. 301, 1982].
\bibitem{b7} M. Young, The Technical Writer's Handbook. Mill Valley, CA: University Science, 1989.
\end{thebibliography}
\vspace{12pt}
\color{red}
IEEE conference templates contain guidance text for composing and formatting conference papers. Please ensure that all template text is removed from your conference paper prior to submission to the conference. Failure to remove the template text from your paper may result in your paper not being published.

\end{document}